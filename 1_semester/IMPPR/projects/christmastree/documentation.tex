\documentclass[10pt,a4paper]{article}
\usepackage[utf8x]{inputenc}
\usepackage{ucs}
\usepackage[english]{babel}
\usepackage{amsmath}
\usepackage{amsfonts}
\usepackage{amssymb}
\usepackage{fancyvrb}
\usepackage{struktex}

\author{Rico Ukro}
\date{\today{}, Kesselsdorf}
\title{Documentation Christmas Tree}

\begin{document}
  \maketitle
  \thispagestyle{empty}
  \begin{figure}
    \centering
    \begin{BVerbatim}
      \ /        
      -X-        
      /|\
       A         
      /X\
     /XXX\
    /XoXXX\
   /XoXX_XX\
  /XXXXXXOXX\
 /XoXXX_XXXXX\
/XoXX*XX_XXXXX\
      |||      
    \end{BVerbatim}
  \end{figure}
  
  \newpage
  \section*{Description}
  \begin{flushleft}
    This program prints out a randomly generated christmas to the linux terminal. \\
    It has been tested under Ubuntu 22.04.
  \end{flushleft}

  \section*{Structograms}
  \begin{center}
  \begin{struktogramm}(110,130)[globals]
    \assign {
      \begin{declaration}[]
        \description{STAR}{star pattern}
        \description{TRUNK}{trunk pattern}
        \description{BACKGROUND\textunderscore ELEMENT}{background element char}
        \description{START\textunderscore CHAR \textunderscore ELEMENT}{tree start element}
        \description{FILLING\textunderscore ELEMENT}{tree filling element}
        \description{STOP\textunderscore CHAR \textunderscore ELEMENT}{tree stop element}
        \description{ORNAMENTS}{holds ornaments}
        \description{ORNAMENT\textunderscore FREQUENCY}{divider to reduce the random occurrence of ornaments}
        \description{RED}{string color red}
        \description{BROWN}{string color brown}
        \description{YELLOW}{string color yellow}
        \description{GREEN}{string color green}
        \description{DARK\textunderscore GREEN}{string color dark green}
        \description{WHITE}{string color white}
        \description{X\textunderscore OFFSET}{horizontal tree offset in terminal}
        \description{Y\textunderscore OFFSET}{vertical tree offset in terminal}
      \end{declaration}
    }
  \end{struktogramm}
  \end{center}
  
  \begin{center}
  \begin{struktogramm}(110,10)[function max]
    \sub{uint max (const uint val1, const uint val2) }
    \return{return val1 \(>\) val2 ? val1 : val2}
  \end{struktogramm}
  \end{center}

  \begin{center}
  \begin{struktogramm}(110,200)[function main] 
    \assign{set \(top\textunderscore height\) to random value}
    \assign{calculate \(top\textunderscore width\)}
    \assign{calculate total \(tree\textunderscore width\)}
    \assign{calculate total \(tree\textunderscore height\)}
    \assign{set \(top\textunderscore start\)}
    \assign{set \(trunk\textunderscore start\)}
    \assign{allocate memory for \(row\)}
    \assign{allocate memory for \(last\textunderscore row\)}
    \assign{set terminal color to \(WHITE\)}

    % print y offset
    \forallin{for 0 to Y\textunderscore OFFSET}
      \forallin{for 0 to tree\textunderscore width}
        \assign{OUTPUT BACKGROUND\textunderscore ELEMENT}
      \forallinend
      \assign{OUTPUT new line}
    \forallinend
    
    % print star
    \forallin{for Y\textunderscore OFFSET to top\textunderscore start}
      \assign{calculate space\textunderscore count}
      \forallin{0 to tree\textunderscore width}
        \ifthenelse{6}{3}
          {before or after star fields?}{\sTrue}{\sFalse}
          \assign{OUTPUT BACKGROUND\textunderscore ELEMENT}
        \change
          \assign{OUTPUT STAR element}
        \ifend
      \forallinend
      \assign{OUTPUT new line}
    \forallinend
    
    % print top
    \forallin{for top\textunderscore start to trunk\textunderscore start}
      \assign{calculate line\textunderscore tree\textunderscore width}
      \assign{calculate space\textunderscore count}
      \ifthenelse{6}{2} 
        {not in first row of top?}{\sTrue}{\sFalse}
        \assign{last\textunderscore row = row}
        \change
      \ifend
      \forallin{0 to tree\textunderscore width}
        \assign{OUTPUT colored tree element}
        \assign{assign tree element to row}
      \forallinend
      \assign{OUTPUT new line}
    \forallinend
  \end{struktogramm}
  \end{center}
  
  \begin{center}
  \begin{struktogramm}(110,100)[function main \textit{continued}] 
    % print trunk
    \forallin{for trunk\textunderscore start to tree\textunderscore height - Y\textunderscore OFFSET}
      \assign{calculate space\textunderscore count}
      \forallin{for 0 to tree\textunderscore width}
        \ifthenelse{6}{3}
        {before or after trunk fields?}{\sTrue}{\sFalse}
          \assign{OUTPUT BACKGROUND\textunderscore ELEMENT}
        \change
          \assign{OUTPUT TRUNK element}
        \ifend
      \forallinend
      \assign{OUTPUT new line}
    \forallinend

    % print y offset
    \forallin{for tree\textunderscore height - y\textunderscore offset to tree\textunderscore height}
      \forallin{for 0 to tree\textunderscore width}
        \assign{OUTPUT BACKGROUND\textunderscore ELEMENT}
      \forallinend
      \assign{OUTPUT new line}
    \forallinend
    \assign{free allocated memory of row}
    \assign{free allocated memory of last\textunderscore row}
  \end{struktogramm}
  \end{center}

  \section*{Future Improvements}
  \begin{itemize}
    \item Animated ornaments in tree
    \item Add background music 
    \item Find out, how to get a fancier yellow color for the star
  \end{itemize}
  \section*{Personal summary}
      \begin{flushleft}
        I learned a lot about programming in ANSI-C. Also this is my first \LaTeX-document. \\
        Thanks to Professor Hara for this nice exercise.
      \end{flushleft}
\end{document}
